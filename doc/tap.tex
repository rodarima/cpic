\documentclass{article}

\usepackage[margin=1cm]{geometry}

%\beamertemplatenavigationsymbolsempty
%\setbeamerfont{page number in head/foot}{size=\small}
%\setbeamertemplate{footline}[frame number]
%\usefonttheme[onlymath]{serif}

\usepackage[utf8]{inputenc}

\usepackage[T1]{fontenc}
\usepackage[english]{babel}
%\usepackage[utf8]{inputenc}

\usepackage{tikz}
\usetikzlibrary{calc}
\usetikzlibrary{arrows, backgrounds}
\usetikzlibrary{matrix, arrows.meta}
\usetikzlibrary{decorations.pathreplacing}
\usetikzlibrary{positioning, fit, shapes}
\usetikzlibrary{external}

\tikzexternalize[prefix=tikz/]

\newcommand{\backupbegin}{
   \newcounter{finalframe}
   \setcounter{finalframe}{\value{framenumber}}
}
\newcommand{\backupend}{
   \setcounter{framenumber}{\value{finalframe}}
}

% Scale all pgf plots, not only coordinates
\usepackage{adjustbox}

%\usepackage{amsfonts}
%\usepackage{amsmath}
%\usepackage{amsthm}
\usepackage{bm}

\usepackage{pgfplots}
\pgfplotsset{compat=1.16}
\usepgfplotslibrary{groupplots}

% Used for split environment
\usepackage{amsmath}
% Separate rows in align environment by this amount
\addtolength{\jot}{1em}

\usepackage{graphicx}
\graphicspath{{../fig/} {fig/}}
\usepackage{subfig}
\usepackage{wrapfig}

% Clickable links
\usepackage{hyperref}
\hypersetup{
	colorlinks,
	citecolor=black,
	filecolor=black,
	linkcolor=black,
	urlcolor=black
}

\usepackage[pdf]{graphviz}

%\usepackage[dvipsnames]{xcolor}
\usepackage{listings}

\lstset{
	language=c,
  basicstyle=\small\ttfamily,  % the size of the fonts that are used for the code
	numbers=none,                   % where to put the line-numbers
  inputencoding=latin1,
  numberstyle=\tiny,  % the style that is used for the line-numbers
  stepnumber=1,                   % the step between two line-numbers. If it's
				    %1, each line 
                                  % will be numbered
  %numbersep=5pt,                  % how far the line-numbers are from the code
  backgroundcolor=\color{white},      % choose the background color.
  showspaces=false,               % show spaces adding particular underscores
  showstringspaces=false,         % underline spaces within strings
  showtabs=false,      % show tabs within strings adding particular underscores
	frame=none,                   % adds a frame around the code
  rulecolor=\color{black},        % if not set, the frame-color may be changed
				   % on line-breaks within not-black text (e.g.
				   % comments (green here))
	tabsize=6,                      % sets default tabsize to 2 spaces
  columns=fullflexible,
  extendedchars=true,
  captionpos=b,                   % sets the caption-position to bottom
  breaklines=true,                % sets automatic line breaking
  breakatwhitespace=false,        % sets if automatic breaks should only happen
				    %at whitespace
  title=\lstname,                   % show the filename of files included with
				    %\lstinputlisting;
                                  % also try caption instead of title
  keywordstyle=\color{blue},          % keyword style
	commentstyle=\color{gray},       % comment style
	stringstyle=\color{brown},         % string literal style
	escapeinside={|*}{*|},            % if you want to add LaTeX within your code
	morecomment=[l][\color{purple}]{\#},
	moredelim=[il][\color{purple}]{@},
	abovecaptionskip=-15pt
%	belowcaptionskip=-15pt
}


\usepackage{csquotes}

\usepackage{siunitx}

\usepackage{multicol}

\usepackage{epigraph}
\setlength{\epigraphwidth}{0.7\textwidth}

\usepackage{caption}
\captionsetup{font=footnotesize}

% Macros para ayudar a la redacción
% Vector
\newcommand*\mat[1]{ \begin{pmatrix} #1 \end{pmatrix}}
\newcommand*\arr[1]{ \begin{bmatrix} #1 \end{bmatrix}}
\newcommand*\V[1]{\bm{#1}}
\newcommand{\E}{\V{E}}
\newcommand{\rhog}{\rho_\text{ghost}}
\newcommand{\F}{\V{F}}
\newcommand{\B}{\V{B}}
\renewcommand*{\v}{\V{v}}
\newcommand{\x}{\V{x}}
\newcommand{\dt}{\Delta t}
\newcommand{\dx}{\Delta x}
\newcommand*\neigh[1]{\mathcal{N}(#1)}

% Norm
\newcommand\norm[1]{\left\lVert#1\right\rVert}

%\title{Particle-in-cell plasma simulation\\with OmpSs-2}
%\author{Rodrigo Arias Mallo\\
%\vspace{1em}
%{\footnotesize Director: Vicenç Beltran Querol \\
%Tutor: Jordi Torres Viñals}}
%\institute{Universitat Politècnica de Catalunya (UPC)}
%\date{\today}

\begin{document}

%\begin{figure}[h]
\begin{tikzpicture}[
		>=latex,
	]
	\matrix[
		nodes={draw,circle,inner sep=0mm,minimum size=8mm},
		column sep=5mm] (t)
	{
		\node (t0) {$t_0$}; &
		\node (t1) {$t_1$}; &
		\node (t2) {$t_2$}; &
		\node (t3) {$t_3$}; \\
	};

	\node[draw,fit=(t0) (t3),inner sep=1mm,rounded corners=4mm] (master) {};
	%\node[draw,fit=(t0) (t3),inner sep=0.5mm,ellipse] (master) {};

	\node[left=1mm of master] {Master process};

	% Draw the workers

	\matrix[
		nodes={
			double,
			double distance=1mm,
			draw,
			circle,
			inner sep=0mm,
			minimum size=8mm
		},
		column sep=5mm,
		right=2cm of t] (w) {
		\node (w0) {$w_0$}; &
		\node (w1) {$w_1$}; &
		\node (w2) {$w_2$}; &
		\node (w3) {$w_3$}; \\
	};

	\node[right=1mm of w] {Workers};

	\coordinate (mid) at ($(t)!0.5!(w)$);

	\matrix[
		matrix of nodes,
		nodes={
			draw,
			rectangle,
			inner sep=0mm,
			minimum size=6mm
		},
		above=4cm of mid,
		column sep=5mm
	] (c)
	{
		\node (c0) {$c_0$}; &
		\node (c1) {$c_1$}; &
		\node (c2) {$c_2$}; &
		\node (c3) {$c_3$}; \\
	};

	\node[above=1mm of c] {CPUs};

	% Draw arrows from threads to cpus

	\foreach \i in {0,...,3}
	{
		%\draw[->,dashed] (t\i) to[out=90,in=-90] (c\i);
		\draw[->,dashed] (t\i) -- (c\i);
	}

	% Draw shared memory

	\matrix[
		nodes={
			anchor=center,
			draw,
			rectangle,
			minimum width=4cm,
			minimum height=1cm,
		},
		row sep =-\pgflinewidth,
		column sep = -\pgflinewidth,
		below=3cm of mid,
		column sep=0mm
	] (m)
	{
		\node (m0) {$m_0$}; \\
		\node (m1) {$m_1$}; \\
		\node (m2) {$m_2$}; \\
		\node (m3) {$m_3$}; \\
	};

	\node[below=1mm of m] {Shared memory};

	% Draw shared memory lines

	\foreach \i in {0,...,3}
	{
		\draw[dashed] (m\i) to[out=0,in=-90] (w\i);
	}

	% Brace fo memory for the master

	\draw [decorate,decoration={brace,mirror,amplitude=5mm},yshift=0pt]
	(m0.north west) -- (m3.south west) node[black,midway,xshift=-5mm] (mem-brace) 
	{};

	\draw[dashed] (mem-brace) to[out=180,in=-90] (t.south);

	% Label processes

	\node[above=1cm of w0] (process) {Processes};

	\draw[->,shorten >=1.5mm] (process) -- (t3);
	\draw[->,shorten >=0.5mm] (process) -- (w0);
	\draw[->,shorten >=0.5mm] (process) -- (w1);

	% Label threads
	\node[below=2cm of mid] (thread) {Threads};
	\draw[->,shorten >=-0.5mm] (thread) -- (w0);
	\draw[->,shorten >=-0.5mm] (thread) -- (t2);
	\draw[->,shorten >=-0.5mm] (thread) -- (t3);

\end{tikzpicture}
%\end{figure}

\end{document}
