\chapter{Analysis of performance}

The execution time is measured and analyzed in detail.

\section{Variables of interest}

A simple test to determine the performance of the simulator is to measured the
time per iteration (using the wall clock). The time spent in the different
stages of the simulation can also be of interest.

Consider the sequence of time per iteration (also referred simply as the time)
$T_0,\ldots,T_n$ as independent random variables from a common distribution with
an unknown mean $\mu$ and finite standard deviation $\sigma$. The sample mean
$\overline T$ can be approximated with a certain degree of confidence by a
process of sampling.

Different parameters of the simulation may affect the time mean $\mu$ and is the
objective of this chapter to find out what the relation is. The time is at least
in $O(N_p)$, as we need to iterate over each particle $N_p$. We also know that
the worst-case complexity of the FFT is in $O(N_g \log N_g)$ for $N_g$ total grid
points.

\section{Number of particles}

The number of particles $N_p$ is one of the main parameters that affect the
running time of each iteration. In order to characterize the scalability of the
application, several tests are designed with increasing number of particles, and
different metrics are compared.

The wall clock is used in the process with rank zero to measure how long takes
each iteration.

\begin{figure}[h]
\centering
\begin{tikzpicture}
\begin{axis}[
	width=0.5\textwidth,
	xlabel=Particles $N_p$,
	ylabel=Time per iteration (s),
	grid=major]
\addplot [only marks,mark=none,error bars/y dir=both, error bars/y explicit] 
table [
	x index = {0},
	y index = {3},
	y error index={4},
	col sep=space] {csv/time.csv};
\end{axis}
\end{tikzpicture}
\begin{tikzpicture}
\begin{axis} [
	grid=major,
	xlabel=Grid points $N_g$,
	ylabel=Time per iteration (s),
	width=0.5\textwidth]
\addplot [only marks,mark=none,error bars/y dir=both, error bars/y explicit] 
table [
	x index = {0},
	y index = {3},
	y error index={4},
	col sep=space] {csv/solver.csv};
\end{axis}
\end{tikzpicture}
\end{figure}
