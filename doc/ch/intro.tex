\chapter{Introduction}
\label{ch:intro}

It may be surprising to find out that the most common state of matter is plasma, 
when we look at the universe. The simulation of plasma has been increasingly 
researched since the computers began gaining computation speed, as it is quite 
complex and expensive to study in a physics laboratory. The particle-in-cell 
methods are now widely used for the simulation of different plasma phenomena, as 
they provide a good parallelization that can be exploited with today 
supercomputers.

\section{Motivation}

The design and implementation of a plasma simulator is a useful way to find out 
the patterns and complexities of a parallel application used in the real world.  
Most of the existing PIC codes are highly tied to solve a specific set of 
simulations to work on some experiments and the documentation is usually poor or 
inexistent, and the designs are the result of years added features without a 
clear design.

Using OmpSs-2 from the beginning can lead to a higher performance execution, and 
at the same time keep a clean and documented design.

\section{Objectives}

One of the main objectives of the simulation is the use of the data-flow 
execution model provided by OmpSs-2 to find the challenging computational 
patterns that occur in a complete and real application.

Furthermore the Task Aware MPI library (TAMPI), will be compared against MPI to 
measure the performance in communications on a complex simulation scenario.

The challenges found during the design of the simulator will be used to improve 
the current solutions provided by the programming model and propose new 
alternatives.

\section{Structure}
%
\begin{figure}[h]%{{{
\centering
\scalebox{0.7} {
\begin{tikzpicture}[>=latex,thick]
	\matrix (m) [
		matrix of nodes,
		column sep=5mm,
		row sep=5mm,
		nodes={
			draw, % General options for all nodes
			line width=1pt,
			anchor=center,
			text centered,
			rounded corners,
			minimum width=5cm,
			minimum height=8mm,
		},
		txt/.style={text width=1.5cm,anchor=center},
	]
	{
		Physical phenomenon \\
		Mathematical model \\
		Discretization \\
		Numerical algorithms \\
		|[fill=black!10]| Parallelization \\
		Simulation program \\
		Computer experiment \\
	};
	\foreach \i [evaluate={\j=int(\i+1)}] in {1,...,6}{
		\draw[->] (m-\i-1) -- (m-\j-1);
	}
	\draw [
		decorate,decoration={brace,amplitude=5pt,raise=10pt},
	] (m-1-1.north east) -- (m-2-1.south east) node 
	[black,midway,right,xshift=18pt] {Chapter~\ref{ch:plasma-sim}};
	\draw [
		decorate,decoration={brace,amplitude=5pt,raise=10pt},
	] (m-3-1.north east) -- (m-4-1.south east) node 
	[black,midway,right,xshift=18pt] {Chapter~\ref{ch:discrete-model}};
	\draw [
		decorate,decoration={brace,amplitude=5pt,raise=10pt},
	] (m-5-1.north east) -- (m-5-1.south east) node 
	[black,midway,right,xshift=18pt]
		{Chapters~\ref{ch:parallel-simulator} and~\ref{ch:comm}};

\end{tikzpicture}
}
\caption{Principal steps in a computer simulation experiment}
\label{fig:structure}
\end{figure}%}}}
%
The structure of the document follows the diagram shown in the 
figure~\ref{fig:structure}. In the chapter~\ref{ch:plasma-sim}, plasma is 
described as a physical phenomenon and we focus on the relevant properties that 
we want to study, from which we derive a mathematical model.  The discretization 
of the model allows the computer simulation by using numerical algorithms, and 
is discussed in the chapter~\ref{ch:discrete-model}. A sequential prototype is 
designed to test the proposed model in chapter~\ref{ch:sequential}.  Then, 
following the techniques described in the chapter~\ref{ch:techniques} a parallel 
simulator is build in chapters~\ref{ch:parallel-simulator} and~\ref{ch:comm}.

Finally, the performance of the simulator is addressed in the 
chapter~\ref{ch:analysis}, leading to the conclusions and future work in the 
chapter~\ref{ch:discussion}.
